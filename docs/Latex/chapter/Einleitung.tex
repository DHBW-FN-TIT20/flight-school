\section{Einleitung}
\subsection{Aufgabenstellung}
Die Prüfungsleistung der Vorlesung Computergrafik beinhaltet die Erstellung eines Programmentwurfs.
\newline
Dieser Programmentwurf besteht aus der Erstellung einer animierten 3D-Computergrafik.
Hierzu sollen HTML, CSS JavaScript und WebGLv2 verwendet werden. Zur Szenenmodellierung darf außerdem die three.js Bibliothek verwendet werden.
Der Programmentwurf muss folgeden Punkte enthalten:
\begin{itemize}
\item Szene ist dreideminensional
\item Einzelne Objekte in der Szene sind animiert
\item Kamera kann sich durch die Szene bewegen
\item Mindestens eine Lichtquelle mit Phong-Beleuchtungmodell
\item Control Panel zur Steuerung der 3D-Grafik
\end{itemize}

\subsection{Aufbau der Arbeit}
Im folgeden werden zunächst die verwendeten Hilfsmittel erläutert,
im Anschluss wird ein Konzept für die 3D-Szenze erarbeitet und in verschiedenen Diagrammen dargestellt.
Abschließend wird das finale Produkt dargestellt und eine Installationanleitung zur Verfügung gestellt.
