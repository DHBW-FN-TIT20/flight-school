\section{Realisierung}
\subsection{Klassenzimmer}
% Exportierung der Blender Modelle in  GTLF Blender
Zunächst werden alle Modelle aus Blender in das GL Transmission Format (.glb) exportiert.
% Einbindung der Modelle in ThreeJs
% Platzierung der Modelle gemäß der Zeichnung aus Abb.
Anschließend werden diese Modelle in three.js eingebunden und gemäß dem Plan in Abbildung \ref{fig:KlassenzimmerEntwurf}
platziert.
Abschließend müssen die Scheiben in die Fenster eingefügt werden, da dies nicht mit einem Export aus Blender möglich ist.
\newparagraph
% Bewegungen / Kollisionserkennungen / sonst. Interaktionen
Nachdem die Szene vollständig erstellt wurde, müssen nachfolgend die definierten Interaktionen hinzugefügt werden.
Um sich im Raum zu bewegen, wird eine unsichtbarer Quader hinzugefügt, der die Person darstellt. Dieser Quader enthält auch die Kamera
und kann mit W A S D durch den Raum bewegt werden. Um Kollisionsen zu erkennen wird vor Ausführung der Bewegung überprüft,
ob die Bewegung ausgeführt werden darf, würde der Quader innerhalb eines anderen Objekts sein, wird die Bewegung nicht ausgeführt.
\newparagraph
Damit die Tafel nach oben bzw. nach unten bewegt werden kann, wird die Tafel mit der Maus angewählt und verschoben.
Hierzu wir bei einem Mausklick überprüft ob auf die Tafel geklickt wurde und anschließend abhängig von der Maus Bewegung in die X-Achsen Richtung die Tafel verschoben.
Das Auf- und Abstuhlen der Stühle funktioniert ähnlich. Es wird ebenfalls überprüft ob bei einem Mausklick auf einen Stuhl geklickt wurde und anschließend wird dieser auf- oder abgestuhlt.
Das öffnen oder schließen der Schranktüren funktioniert analog zu den Stühlen.
Zusätzlich darf maximal ein Abstand von vier Metern zwischen dem Objekt und dem Personen-Quader sein um die Animation auszuführen.
% // lukas
\subsection{Flugsimulator}
% ThreeJs Meer + Licht,
% Flugzeugmodell
% orbitcontrols
% platzieren der Ringe
% Kollisionserkennung
% Score

%// johannes